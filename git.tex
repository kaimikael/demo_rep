\documentclass[swedish]{beamer}
\usepackage{pgfpages}
\usepackage{myslides}
\usepackage{fancyvrb}
\usepackage[lighttt]{lmodern}
\usepackage{menukeys}
\lstset{language=bash}

\makeatletter
\AtBeginDocument{% 
   \@ifpackageloaded{fancyvrb}{}{\let\@xpace@fancyvrb@catcodes\relax}% 

} 

\def\@xpace@fancyvrb@catcodes{% 
   \ifx\relax \FV@CommandChars 
   \else 
     \expandafter\@xpace@fancyvrb@catcodes@aux\FV@CommandChars 
   \fi 
} 

\def\@xpace@fancyvrb@catcodes@aux 
   \catcode`#1=0\relax\catcode`#2=1\relax\catcode`#3=2\relax{% 
     \@makeother#1\@makeother#2\@makeother#3% 
} 

\def\@xspace@eTeX@setup{% 
   \begingroup 
   \everyeof{}% 
   \endlinechar=-1\relax 
   \@xpace@fancyvrb@catcodes % extra line here. 
   \catcode`\ =10\relax 
   \makeatletter 
   \catcode`\\\z@ 
   \catcode`\{\@ne 
   \catcode`\}\tw@ 
   \expandafter\scantokens\expandafter{\expandafter\gdef 
     \expandafter\@xspace@exceptions@tlp 
     \expandafter{\@xspace@exceptions@tlp}}% 
   \endgroup 
} 
\makeatother

\newenvironment{dialogue}{%
\VerbatimEnvironment
\begin{Verbatim}[fontsize=\footnotesize,commandchars=\#\(\)]%
}
{%
\end{Verbatim}
}

\title{Versionshantering och git}
\author{Kai-Mikael Jää-Aro}
\date{}


\begin{document}
\setlength{\intextsep}{0mm}
%\setbeameroption{show notes on second screen=left}

\begin{frame}
\titlepage
\end{frame}

\begin{frame}
  \frametitle{Versionskontroll}
Mjukvaruutveckling är inte en linjär process från en tom fil till \(n\) rader korrekt fungerande kod.
\begin{itemize}
\item Fel införs under arbetets gång.
\item Vi provar alternativa lösningar.
\item Flera parallella versioner krävs.
\end{itemize}

Därför måste vi:
\begin{itemize}
\item kunna återgå till en fungerande version av mjukvaran,
\item arbeta med en av flera möjliga versioner.
\end{itemize}

En lösning på detta problem är \emph{versionskontroll}.
\end{frame}

\begin{frame}
  \begin{itemize}
  \item Ett arkiv
  \item\emph{Incheckning} av kod (\emph{revisioner}) i arkivet.
\note[item]{Incheckningen ska också ha kommentar om vad som ändrats sen sist.}
  \item  Trädstruktur för parallella versioner 
  \item Hämta ut tidigare incheckningar
  \end{itemize}
\end{frame}

\begin{frame}
\includegraphics[height=0.8\textheight]{revision-tree}
\end{frame}

\begin{frame}
``tip''/``trunk''/``master'' är huvudspåret.

Varje revision har en identitet, kan markeras med en \emph{tag} som 
motsvarar släppt version eller annan milsten.

Ändringar i parallella versioner kan föras tillbaka till master.
\end{frame}

\begin{frame}
\frametitle{Parallell utveckling}

Flera utvecklare i samma projekt.
\begin{enumerate}
\item Checka ut aktuell version av koden.
\item Koda, testa.
\item Checka in \emph{stabil} kod.
\item Hantera ev konflikter.
\item Repetera tills projektet klart.
\end{enumerate}
\end{frame}

\begin{frame}[fragile]
  \frametitle{Konflikthantering}
Antag att två kodare checkar in varsin version av metoden \lstinline+func1+:
\begin{tabular}{ll}
Kodare 1&Kodare 2\\
\begin{minipage}[t]{0.45\linewidth}
\begin{lstlisting}[basicstyle=\tiny]
public int func1(int par) {
  int var1 = 0;  // Ursprunglig rad
  var1 = par;   // Ny rad
  return(var1);
}
\end{lstlisting}
\end{minipage}
&
\begin{minipage}[t]{0.45\linewidth}
\begin{lstlisting}[basicstyle=\tiny]
public int func1(int par) {
  return(par);
}
\end{lstlisting}
\end{minipage}
\end{tabular}
\end{frame}

\begin{frame}[fragile]
Raden \lstinline+var1 = par;+ läggs till.
\lstinline+int var1 = 0;+ tas
bort.  

\lstinline+return+-satsen skiljer sig mellan de två incheckningarna, en \emph{konflikt}.  Denna måste lösas manuellt.

\end{frame}

\begin{frame}[fragile]
Versionshanteringssystemet förstår bara koden som en sträng text.
Överlappande ändringar i samma kod flaggas, men om någon skulle ändra
funktionshuvudet till \lstinline+public void func1(int par1, int
par2)+ invalideras alla anrop till \lstinline+func1+, men
versionshanteringssystemet ser inte detta.

Därför bör alla incheckningar följas av ett \emph{bygge}.
\end{frame}

\begin{frame}
\frametitle{Git}
Det finns ett stort antal versionshanteringssystem, här kommer vi att gå igenom git.

  git är ett \emph{distribuerat} versionskontrollsystem.  \Mao lagrar \emph{alla} utvecklare en (komprimerad) kopia av hela databasen, men normalt utser man ett specifikt arkiv till att vara ``origin'', till vilket alla kopierar sitt material och som alla hämtar ifrån.

Dock är man inte beroende av det centrala arkivet utan kan arbeta offline med sin kopia tills man får kontakt med arkivet igen.

\end{frame}

\begin{frame}
  \frametitle{Att använda git}
Grunden för git är ett kommndoradsgränssnitt, men det finns flera olika grafiska gränssnitt.

GitHub (\url{https://github.com/}) gör tillgängligt både ett webbgränssnitt till git och gratis server-utrymme för open source-projekt.

\end{frame}


\begin{frame}
\frametitle{Att skapa ett arkiv}

\end{frame}

\begin{frame}[fragile]
\end{frame}


\begin{frame}[fragile]
\end{frame}

\begin{frame}[fragile]
\frametitle{Att skapa ett repository}  
\begin{columns}
\begin{column}{0.5\textwidth}
Att skapa ett lokalt arkiv är lätt: gå till aktuell katalog och skriv:
\begin{dialogue}
prompt> #textbf(git init)
\end{dialogue}
Detta skapar en underkatalog \directory{.git}, som kommer att innehålla det faktiska arkivet.
\end{column}
\begin{column}{0.5\textwidth}
\menu{Repository > Add or Create\dots}

\vspace{\baselineskip}

\includegraphics[width=\textwidth]{CreateRepository}
\end{column}
\end{columns}
\end{frame}

\begin{frame}[fragile]
\frametitle{Att förbereda filer för lagring (\emph{staging})}
\begin{columns}
\begin{column}{0.5\textwidth}
Man måste säga till att filer ska lagras i arkivet. 
\begin{dialogue}
prompt> #textbf(git add #textsl(filnamn))
\end{dialogue}

Filen/filerna kommer nu att placeras i ett index \emph{i aktuellt skick}, men ännu inte läggas i arkivet.
\end{column}
\begin{column}{0.5\textwidth}
\menu{Stage}

\vspace{\baselineskip}

\includegraphics[width=\textwidth]{StageFiles}

\end{column}
\end{columns}
\end{frame}

\begin{frame}[fragile]
\frametitle{Att lagra filer (\emph{commit})}  
\begin{columns}
\begin{column}{0.5\textwidth}
För att faktiskt placera filerna i arkivet gör man 
\begin{dialogue}
prompt> #textbf(git commit)
\end{dialogue}

Om man tidigare har adderat filer till indexet, så kommer git att följa dem även i framtiden och man kan placera alla följda filer i arkivet med
\begin{dialogue}
prompt #textbf(git commit -a)
\end{dialogue}
\end{column}
% Commit-meddelanden med biljettnummer från tracking-systemet
\begin{column}{0.5\textwidth}
\menu{Commit}

\vspace{\baselineskip}

\includegraphics[width=\textwidth]{CommitFiles}

\end{column}
\end{columns}
\end{frame}

\begin{frame}[fragile]
\frametitle{Filer som inte behöver lagras}  
Normalt är man bara intresserad av att arkivera \emph{källkodsfiler} (till vilka vi räknar texturer, bilder, animationer \odyl), men inte temporära filer, kompilerad kod, \osv.  För att undvika dessa kan man skapa en fil \directory{.gitignore} med reguljära uttryck som anger vilka filer som \emph{inte} ska lagras.  (Det finns \href{https://github.com/github/gitignore}{färdiga \directory{.gitignore}-filer} för ett antal vanliga utvecklingssystem, som \tex Unity.)

\lstinputlisting[basicstyle=\tiny]{Unity.gitignore}
\end{frame}

\begin{frame}[fragile]
\frametitle{Att se vad som hänt i filerna}  
\begin{columns}
\begin{column}{0.5\textwidth}
Över tid ansamlar sig utvecklingshistorien i arkivet.  Man kan studera historien med 
\begin{dialogue}
prompt> #textbf(git log)
\end{dialogue}
Därför bör \emph{commit}-meddelandena vara koncisa och meningsfulla.

\end{column}
\begin{column}{0.5\textwidth}
\menu{Log}

\vspace{\baselineskip}

\includegraphics[width=\textwidth]{Log}
\end{column}
\end{columns}
\end{frame}

\begin{frame}[fragile]
Man kan jämföra olika versioner av en given fil:
\begin{dialogue}
prompt> #textbf(git diff Repairer.cs)
diff --git a/Assets/Scripts/Repairer.cs b/Assets/Scripts/Repairer.cs
index 879f759..b537641 100644
--- a/Assets/Scripts/Repairer.cs
+++ b/Assets/Scripts/Repairer.cs
@@ -33,7 +33,7 @@ public class Repairer : NPC {
     public override bool OnDroppedItem(Item item)
     {
         if (item == Item.wrench)
-        {//om rosen som släpptes på denna NPC är en ros
+        {//om objektet som släpptes på denna NPC är en skiftnyckel^M
             //Apperance, knowledge, reputation, confidence
             player.changeStats#char"28()0, 0, 1, 1#char"29;
             Say#char"28"Thanks, now I can fix this coffeemachine!"#char"29;
\end{dialogue}
\end{frame}

\begin{frame}
      \includegraphics[width=\textwidth]{Diff}
\end{frame}

\begin{frame}[fragile]
\frametitle{Att återställa en tidigare version}
Om (när) man inser att man behöver återgå till en tidigare version kan man hämta denna med
\begin{dialogue}
prompt> #textbf(git checkout #textit(commitnr) #textit(filnamn))
\end{dialogue}  
\end{frame}

\begin{frame}[fragile]
\begin{columns}
\begin{column}{0.5\textwidth}
Välj en fil i loggen, högerklicka \menu{Show Changes}.

\vspace{\baselineskip}

\includegraphics[width=\textwidth]{UpdatedLog}
\end{column}
\begin{column}{0.5\textwidth}
\menu{Take Left}/\menu{Take Right}

\vspace{\baselineskip}

\includegraphics[width=\textwidth]{FileCompare}
\end{column}
\end{columns}
\end{frame}

\begin{frame}[fragile]
\frametitle{Att hämta från ett centralt arkiv}
\begin{columns}[t]
\begin{column}{0.5\textwidth}
Att ha ett centralt arkiv förutsätter tillgång till den aktuella servern (mattis, alias 193.11.44.151).  I det här fallet är Kai-Mikael administratören, be honom skapa de arkiv ni behöver.

\vspace{4\baselineskip}

Detta arkiv kan sedan laddas ner med 
\begin{dialogue}
prompt> #textbf(git clone git@193.11.44.151:#textsl(arkivnamn).git)
\end{dialogue}
Vi får då en kopia av hela det angivna arkivet.
\end{column}
\begin{column}{0.5\textwidth}
\menu{Repository > Clone\dots}

\vspace{\baselineskip}

\includegraphics[width=\textwidth]{Clone}
\end{column}
\end{columns}
\end{frame}

\begin{frame}[fragile]
\begin{columns}
\begin{column}{0.5\textwidth}
Om man redan har ett ett projekt utan versionkontroll kan man ladda ner ett tomt arkiv och sedan kopiera över dess \directory{.git}-katalog till den katalog man vill använda (eller vice versa), addera filerna i den och göra commit.  För därefter över filerna till centralarkivet med
\begin{dialogue}
prompt> #textbf(git push origin master)
\end{dialogue}
\end{column}
\begin{column}{0.5\textwidth}
\menu{Commit}

\vspace{\baselineskip}

\includegraphics[width=\textwidth]{CommitAndPush}
\end{column}
\end{columns}
\end{frame}

\begin{frame}[fragile]
\frametitle{Att hämta från ett  \emph{remote repository}}
Att sedan hämta uppdaterade filer från det centrala arkivet görs med 
\begin{dialogue}
prompt> #textbf(git fetch)
\end{dialogue}

%Nu kan dessa filer dock ha \emph{konflikter} med filer man själv har ändrat.
%git pull
\end{frame}

% \begin{frame}
% \frametitle{Att döpa ett \emph{remote repository}}
% git remote add
% \end{frame}


% \begin{frame}
% \frametitle{Att lagra på ett  \emph{remote repository}}
% git push  
% \end{frame}

% \begin{frame}
% \frametitle{\emph{Tags} och \emph{branches}}
% git tag  
% git push <tagname>

% git branch <branchname>
% git checkout <branchname>
% \end{frame}

\begin{frame}[fragile]
\frametitle{Att ha en privat utvecklingsgren}
Ett vanligt arbetsflöde är att man har en \emph{master branch}, där koden är garanterad att vara stabil och körbar.  Olika utvecklare kan sedan lätt skapa en egen gren att koda och testa i.  Om något blir fel kan man alltid backa i denna gren och hålla mastern stabil.  Man kan självfallet göra ytterligare grenar för olika tester.

git-arkiven är uppbyggda så att de låter oss arbeta på vilken punkt i trädet vi vill.  Man kan se det som att man har en pekare in i trädet som pekar ut den aktuella uppsättningen filer som vi arbetar på.  Vid behov kan vi flytta pekaren nån annanstans och arbeta där istället.
\end{frame}

\begin{frame}[fragile]
  \begin{columns}
    \begin{column}{0.5\textwidth}
      Vi skapar en branch med 
      \begin{dialogue}
prompt> #textbf(git branch #textit(branchnamn))
      \end{dialogue}
och säger att vi vill arbeta med en specific branch
\begin{dialogue}
prompt> #textbf(git checkout #textit(branchnamn))  
\end{dialogue}
Vi hamnar då i ''spetsen'' av den aktuella grenen.
    \end{column}
    \begin{column}{0.5\textwidth}
\menu{Add Branch}

\vspace{\baselineskip}

\includegraphics[width=\textwidth]{BranchAndCheckout}
    \end{column}
  \end{columns}
  
\end{frame}

\begin{frame}[fragile]
\frametitle{\emph{Merge}}
\begin{columns}
\begin{column}{0.5\textwidth}
När man har en utvecklingsversion man är nöjd med är det dags att koppla ihop den med mastern.  
\begin{dialogue}
prompt> #textbf(git checkout master)
prompt> #textbf(git merge otherbranch)
\end{dialogue}

git kommer nu att försöka kombinera filerna från bägge grenarna.
\end{column}
\begin{column}{0.5\textwidth}
\menu{Merge}

\vspace{\baselineskip}

\includegraphics[width=\textwidth]{Merge}

\end{column}
\end{columns}
\end{frame}

\begin{frame}[fragile]
    Det kan vara så att det finns konflikter (överlappande ändringar) i koden, så att man måste gå igenom manuellt och välja vilket av de möjliga alternativen som ska användas.  Då kan man ta hjälp av 
\begin{dialogue}
prompt> #textbf(git mergetool)
\end{dialogue}
(kan kräva installation av lämplig mjukvara).

Om man inser att man gjort bort sig fullständigt, kan man ångra sig:
\begin{dialogue}
prompt> #textbf(git merge --abort)
\end{dialogue}

Till sist, när allt är OK, så gör man en commit.
\end{frame}
\begin{frame}[fragile]
\frametitle{\emph{merge} av binära filer}
git klarar inte (idag) av att göra merge på binära filer -- \mao texturer, animationer, \odyl.  Somliga installationer kan ha en mergetool som kan hjälpa till, men i allmänhet måste man manuellt hämta de binära filer man vill ha med 
\begin{dialogue}
prompt> #textbf(git checkout #textit(treeish) #textit(filnamn))  
\end{dialogue}
  
\end{frame}


\begin{frame}[fragile]
\frametitle{Mer läsning}  
\href{http://git-scm.com/book/en/v2/}{\textsl{Pro Git, 2nd Edition}, Scott Chacon \& Ben Straub.}
\begin{dialogue}
prompt> #textbf(man git)
\end{dialogue}
\end{frame}
\end{document}
